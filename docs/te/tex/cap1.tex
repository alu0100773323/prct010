%%%%%%%%%%%%%%%%%%%%%%%%%%%%%%%%%%%%%%%%%%%%%%%%%%%%%%%%%%%%%%%%%%%%%%%%%%%%%
% Chapter 1: Motivaci�n y Objetivos 
%%%%%%%%%%%%%%%%%%%%%%%%%%%%%%%%%%%%%%%%%%%%%%%%%%%%%%%%%%%%%%%%%%%%%%%%%%%%%%%

Los objetivos le dan al lector las razones por las que se realiz� el
proyecto o trabajo de investigaci�n.

%---------------------------------------------------------------------------------
\section{Secci�n Uno}
\label{1:sec:1}
  Primer p�rrafo de la primera secci�n.
Si simplemente se desea escribir texto normal en Latex
sin complicadas f�rmulas matem�ticas esfectos especiales
como cambios de fuentes, entonces simplemente tienenn que escribir
en espa�ol normalmente.
si desea cambiar de p�rrafo ha de dejar una l�nea en blanco o bien 
utilizar el comando\par
No es necesario preocuparse de la sangr�a de los p�rrafos:
todos los p�rrafos se sangraran autom�ticamente con excepci�n
del primer p�rrafo de una secci�n.\par
Se ha de distinguir entre la comilla simple 'izquierda'
y la comilla simple 'derecha' cuando se escribe en el ordenador.
En el caso de que quieran utilizar las comillas dobles se han de 
escribir dos caracteres 'comillas simples' seguidos, esto es,
''comillas dobles''.

Tambi�n se ha de tener cuidado con los guiones: se utiliza un �nico
gui�n para la separaci�n de s�labas, mientras que se utilizan
tres guiones seguidos para producir un gui�n de los que se usan 
como signo de puntuaci�n --- como en esta oraci�n.


%---------------------------------------------------------------------------------
\section{Secci�n Dos}
\label{1:sec:2}
  Primer p�rrafo de la segunda secci�n.

\begin{itemize}
  \item Item 1
  \item Item 2
  \item Item 3
\end{itemize}

