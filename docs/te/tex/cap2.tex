%%%%%%%%%%%%%%%%%%%%%%%%%%%%%%%%%%%%%%%%%%%%%%%%%%%%%%%%%%%%%%%%%%%%%%%%%%%%%%%
% Chapter 2: Fundamentos Te�ricos 
%%%%%%%%%%%%%%%%%%%%%%%%%%%%%%%%%%%%%%%%%%%%%%%%%%%%%%%%%%%%%%%%%%%%%%%%%%%%%%%

%++++++++++++++++++++++++++++++++++++++++++++++++++++++++++++++++++++++++++++++

En este cap�tulo se han de presentar los antecedentes te�ricos y pr�cticos que
apoyan el tema objeto de la investigaci�n.

%++++++++++++++++++++++++++++++++++++++++++++++++++++++++++++++++++++++++++++++

\section{Primer apartado del segundo cap�tulo}
\label{2:sec:1}
  Primer p�rrafo de la primera secci�n.
En \LaTeX es sencillo escribir expresiones
matem�ticas como $a=\sum_{i=1}^{10} {x_i}^{3}$y 
deben ser escritas entre dos s�mbolos \$.
Los super�ndices se obtienen con el s�mbolo \^{}, y
los sub�ndices con el s�mbolo \_.
Por ejemplo: $x^2 \times y^{\alpha + \beta}$.
Tambi�n se pueden escribir f�rmulas centradas:
\[h^2=a^2 + b^2 \]
\section{Segundo apartado del segundo cap�tulo}
\label{2:sec:2}
  Primer p�rrafo de la segunda secci�n.

